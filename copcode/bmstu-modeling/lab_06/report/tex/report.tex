\documentclass{bmstu}

\begin{document}

\makereporttitle
{Информатика и системы управления (ИУ)}
{Программное обеспечение ЭВМ и информационные технологии (ИУ7)}
{Лабораторной работе №6}
{Моделирование}
{Моделирование работы электронной очереди}
{}
{ИУ7-73Б}
{К.Э. Ковалец}
{И.В. Рудаков}


\setcounter{page}{2}
\renewcommand{\contentsname}{Содержание} 
\tableofcontents

\chapter{Моделируемая модель}

\section{Задание}

В данной лабораторной работе моделируется следующая система. В пункт получения документов приходят клиенты с заданным интервалом времени, которые сначала подходят к терминалу выдачи талонов. У каждого терминала формируется своя очередь. Клиент выбирает очередь с минимальной длиной. Терминалы обслуживают клиентов за заданный интервал времени. Далее клиент отправляется к окну, в котором его обслужат. У каждого окна формируется своя очередь. Клиента отправляют к окну с минимальной очередью. В окне клиента обслуживают за заданный интервал времени. Количество клиентов задается.

\section{Схема модели}

На рисунке \ref{img:scheme} представлена структурная схема модели.

\imgs{scheme}{h!}{0.6}{Структурная схема модели}

\chapter{Результаты работы}

\section{Листинги программы}

В листинге \ref{lst:generator} представлена реализация генератора.

\mylisting[python]{generator.py}{firstline=1,lastline=23}{Реализация генератора}{generator}{}

В листинге \ref{lst:processor} представлена реализация канала обслуживания.

\mylisting[python]{processor.py}{firstline=1,lastline=41}{Реализация канала обслуживания}{processor}{}

В листинге \ref{lst:eventModel} представлена реализация моделирования работы электронной очереди.

\mylisting[python]{eventModel.py}{firstline=1,lastline=57}{Реализация моделирования работы электронной очереди}{eventModel}{}

\clearpage

\section{Демонстрация работы программы}

На рисунке \ref{img:result} представлен пример работы программы.

\imgs{result}{h!}{0.65}{Результат работы программы}

\end{document}
