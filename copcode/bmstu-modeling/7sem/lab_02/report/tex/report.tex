\documentclass{bmstu}

\begin{document}

\makereporttitle
{Информатика и системы управления (ИУ)}
{Программное обеспечение ЭВМ и информационные технологии (ИУ7)}
{Лабораторной работе №2}
{Моделирование}
{Марковские процессы}
{}
{ИУ7-73Б}
{К.Э. Ковалец}
{И.В. Рудаков}


\setcounter{page}{2}
\renewcommand{\contentsname}{Содержание} 
\tableofcontents

\chapter{Задание}

Разработать графический интерфейс, который позволяет по заданной матрице интенсивностей перехода состояний определить время пребывания системы в каждом состоянии в установившемся режиме работы системы. Для каждого состояния также требуется рассчитать предельную вероятность. Количество состояний не более десяти.


\chapter{Теоретическая часть}

\section{Марковский процесс}

Случайный процесс называется марковским процессом, если для каждого момента времени $t$ вероятность любого состояния системы в будущем зависит только от ее состояния в настоящем и не зависит от того, как система пришла в это состояние.

\section{Предельная вероятность}

Для определения предельной вероятности необходимо решить систему уравнений Колмагорва, в которой все производные приравниваются к нулю, а одно из уравнений заменяется на условие нормировки:

\begin{equation}
    \begin{aligned}
        \sum_{j = 1}^{n} p_{j}(t) = 1.
    \end{aligned}
\end{equation}

\section{Точки стабилизации состояния системы}

Для определения точек стабилизации состояния системы нужно определить вероятности нахождения в определённых состояниях с некоторым малым шагом $\Delta t$. В тот момент, когда разница между вычисленной на данном шаге вероятностью и предельной вероятности будет достаточно мала (< $EPS$), то точка стабилизации считается найденной.


\chapter{Результаты работы}

\section{Листинги программы}

В листинге \ref{lst:markovChains} представлен класс $MarkovChains$, отвечающий за определение времени пребывания системы в каждом состоянии в установившемся режиме работы системы и за рассчет предельных вероятностей.

\mylisting[python]{markovChains.py}{firstline=1,lastline=94}{class MarkovChains}{markovChains}{}

\clearpage

\section{Демонстрация работы программы}

На рисунках \ref{img:matrix4} - \ref{img:graph10} представлены примеры работы программы.

\imgs{matrix4}{h!}{0.36}{Система из 4 состояний}

\imgs{graph4}{h!}{0.4}{График вероятности от времени для системы из 4 состояний}

\imgs{matrix10}{h!}{0.4}{Система из 10 состояний}

\imgs{graph10}{h!}{0.45}{График вероятности от времени для системы из 10 состояний}

\end{document}
