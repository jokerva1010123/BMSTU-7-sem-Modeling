\documentclass{bmstu}

\begin{document}

\makereporttitle
{Информатика и системы управления (ИУ)}
{Программное обеспечение ЭВМ и информационные технологии (ИУ7)}
{Лабораторной работе №5}
{Моделирование}
{Моделирование работы информационного центра}
{}
{ИУ7-73Б}
{К.Э. Ковалец}
{И.В. Рудаков}


\setcounter{page}{2}
\renewcommand{\contentsname}{Содержание} 
\tableofcontents

\chapter{Задание}

В информационный центр приходят клиенты через интервалы времени 10±2 минуты. Если все три имеющихся оператора заняты, клиенту отказывают в обслуживании. Операторы имеют разную производительность и могут обеспечивать обслуживание среднее запросы за 20±5, 40±10, 40±20 минут. Клиенты стремятся занять свободного оператора с максимальной производительностью. Полученные запросы сдаются в приемные накопители, откуда они выбираются для обработки. На первый компьютер -- запросы от первого и второго операторов, на второй компьютер -- от третьего оператора. Время обработки на первом и втором компьютере равны соответственно 15 и 30 минутам. Смоделировать процесс обработки 300 запросов. Определить вероятность отказа.

\chapter{Теоретическая часть}

\section{Схемы модели}

На рисунке \ref{img:blockDiagram} представлена структурная схема модели.

\imgs{blockDiagram}{h!}{0.45}{Структурная схема модели}

В процессе взаимодействия клиентов с информационным центром возможно два режима работы:

\begin{itemize}
    \item режим нормального обслуживания, когда клиент выбирает одного из свободных операторов, отдавая предпочтение тому, у кого максимальная производительность;
    \item режим отказа клиенту в обслуживании, когда все операторы заняты.
\end{itemize}

На рисунке \ref{img:queuingSystems} представлена схема модели в терминах систем массового обслуживания (СМО).

\imgs{queuingSystems}{h!}{0.45}{Схема модели в терминах СМО}

\section{Переменные и уравнение имитационной модели}

\textbf{Эндогенные переменные:}

\begin{itemize}
    \item время обработки задания $i$-ым оператором;
    \item время решения задания на $j$-ом компьютере.
\end{itemize}

\textbf{Экзогенные переменные:}

\begin{itemize}
    \item $n0$ — число обслуженных клиентов;
    \item $n1$ — число клиентов, получивших отказ.
\end{itemize}

Вероятность отказа в обслуживании клиента будет вычисляться как:

\begin{equation}
    P = \frac{n_0}{n_0 + n_1}
\end{equation}

\chapter{Результаты работы}

\section{Листинги программы}

В листинге \ref{lst:generator} представлена реализация генератора.

\mylisting[python]{generator.py}{firstline=1,lastline=16}{Реализация генератора}{generator}{}

В листинге \ref{lst:processor} представлена реализация канала обслуживания.

\mylisting[python]{processor.py}{firstline=1,lastline=28}{Реализация канала обслуживания}{processor}{}

В листинге \ref{lst:eventModel} представлена реализация моделирования работы информационного центра.

\mylisting[python]{eventModel.py}{firstline=1,lastline=56}{Реализация моделирования работы информационного центра}{eventModel}{}

\clearpage

\section{Демонстрация работы программы}

На рисунке \ref{img:result} представлен пример работы программы.

\imgs{result}{h!}{0.65}{Результат работы программы}

\end{document}
