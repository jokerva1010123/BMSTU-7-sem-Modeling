\documentclass{bmstu}

\begin{document}

\makereporttitle
{Информатика и системы управления (ИУ)}
{Программное обеспечение ЭВМ и информационные технологии (ИУ7)}
{\textbf{1}}
{Моделирование}
{Функции распределения и плотности распределения}
{4}
{ИУ7-73Б}
{К.Э. Ковалец}
{И.В. Рудаков}


\setcounter{page}{2}
\renewcommand{\contentsname}{Содержание} 
\tableofcontents

\chapter{Задание}

Необходимо разработать программу для построения графиков распределения и плотности распределения для:

\begin{itemize}
    \item равномерного распределения;
    \item распределения Эрланга.
\end{itemize}

Также необходио разработать графический интерфейс для ввода параметров функций распределения.

\chapter{Теоретическая часть}

\section{Равномерное распределение}

Функция равномерного распределения:

\begin{equation}
    F(x) =
    \begin{cases}
            0, x < a, \\
            \begin{aligned}
                \frac{x -  a}{b - a}, x \in [a, b], 
            \end{aligned}\\
            0, x > b. \\
    \end{cases}
\end{equation}

Функция плотности равномерного распределения:

\begin{equation}
    f(x) =
    \begin{cases}
            \begin{aligned}
                \frac{1}{b - a}, x \in [a, b], 
            \end{aligned}\\
            0, else. \\
    \end{cases}
\end{equation}

\section{Распределение Эрланга}

Функция распределения Эрланга:

\begin{equation}
    \begin{aligned}
        F_k(x) = 1 - e^{-\lambda \cdot x} \cdot \sum_{i = 1}^{k - 1} \frac{(\lambda \cdot x)^i}{i!}.
    \end{aligned}
\end{equation}


Функция плотности распределения Эрланга:

\begin{equation}
    \begin{aligned}
        f_k(x) = \frac{\lambda \cdot (\lambda \cdot x)^{k - 1}}{(k - 1)!} \cdot e^{-\lambda \cdot x}.
    \end{aligned}
\end{equation}

В данных формулах $\lambda$ и $k$ --- положительные параметры распределения $(\lambda \geqslant 0; k = 1, 2, ...)$;
$x \geqslant 0$.

\chapter{Результаты работы}

\section{Листинги программы}

В листинге \ref{lst:distribution} представлен класс $Distribution$, отвечающий за вычисление значений функций распределения и плотности распределения.

\mylisting[python]{distribution.py}{firstline=1,lastline=23}{class Distribution}{distribution}{}

\section{Демонстрация работы программы}

На рисунках \ref{img:uniformDistr} - \ref{img:erlangDistr} представлены примеры работы программы.

\imgs{uniformDistr}{h!}{0.33}{Равномерное распределение}

\imgs{erlangDistr}{h!}{0.33}{Распределение Эрланга}

\end{document}
